\chapter{\ifenglish Introduction\else บทนำ\fi}

\section{\ifenglish Project rationale\else ที่มาของโครงงาน\fi}

ปัจจุบัน นักสังคมสงเคราะห์ที่ปฏิบัติงาน ณ โรงพยาบาลมหาราชนครเชียงใหม่ ยังคงใช้วิธีการจัดเก็บข้อมูลในรูปแบบกระดาษ ก่อนจะทำการบันทึกข้อมูลลงในคอมพิวเตอร์ของตนเองในรูปแบบไฟล์ .docx (Microsoft Word) อย่างไรก็ตาม คอมพิวเตอร์แต่ละเครื่องไม่สามารถเข้าถึงไฟล์ของกันและกันได้โดยตรง หากนักสังคมสงเคราะห์ต้องการเข้าถึงข้อมูลของผู้ใช้บริการที่ไม่ได้จัดเก็บอยู่ในคอมพิวเตอร์ของตนเอง พวกเขาจำเป็นต้องสอบถามเพื่อนร่วมงานเพื่อระบุว่าไฟล์ดังกล่าวถูกจัดเก็บอยู่ในเครื่องใด
นอกจากนี้ สำหรับข้อมูลสถิติของผู้รับบริการ นักสังคมสงเคราะห์จะบันทึกข้อมูลลงในไฟล์ .xlsx (Microsoft Excel) ซึ่งสามารถแชร์ร่วมกันระหว่างคอมพิวเตอร์หลายเครื่องได้ อย่างไรก็ตาม การจัดเก็บข้อมูลดังกล่าวดำเนินการในไฟล์เดียว โดยมีจำนวนข้อมูลสะสมหลายพันบรรทัด ส่งผลให้การสรุปผลข้อมูลในระดับรายวันหรือรายเดือนทำได้ยาก และอาจส่งผลต่อประสิทธิภาพในการบริหารจัดการข้อมูลโดยรวม

จากปัญหาที่กล่าวมาข้างต้น คณะผู้จัดทำโครงงานได้ตระหนักถึงข้อจำกัดในการจัดเก็บและบริหารจัดการข้อมูลของนักสังคมสงเคราะห์ จึงมีแนวคิดในการพัฒนา เว็บแอปพลิเคชัน เพื่อช่วยให้กระบวนการจัดการข้อมูลมีประสิทธิภาพมากยิ่งขึ้น ระบบดังกล่าวได้รับการออกแบบมาเพื่อแก้ไขปัญหาด้านการค้นหาข้อมูลให้สะดวกและรวดเร็วขึ้น รวมถึงปรับปรุงกระบวนการสรุปผลข้อมูลให้สามารถดำเนินการได้อย่างเป็นระบบและมีประสิทธิภาพยิ่งขึ้น

\section{\ifenglish Objectives\else วัตถุประสงค์ของโครงงาน\fi}
\begin{enumerate}
    \item เพื่อพัฒนาเว็บแอปพลิเคชันที่สามารถจัดเก็บข้อมูล และเข้าถึงข้อมูลได้อย่างสะดวก รวดเร็ว และมีประสิทธิภาพ
    \item เพื่อพัฒนาเว็บแอปพลิเคชันที่สามารถสรุปผลข้อมูลที่ผู้ใช้ต้องการได้ โดยไม่ต้องใช้โปรแกรมอื่นเพิ่มเติม
\end{enumerate}

\section{\ifenglish Project scope\else ขอบเขตของโครงงาน\fi}

\subsection{\ifenglish Hardware scope\else ขอบเขตด้านฮาร์ดแวร์\fi}
โครงงานนี้เป็นการพัฒนาเว็บแอปพลิเคชันที่สามารถใช้งานได้บนคอมพิวเตอร์ที่ใช้ระบบปฏิบัติการ macOS
หรือ Windows และบนโทรศัพท์มือถือที่ใช้ระบบปฏิบัติการ iOS หรือ Android
\subsection{\ifenglish Software scope\else ขอบเขตด้านซอฟต์แวร์\fi}
\begin{enumerate}
    \item เว็บแอปพลิเคชันที่สามารถเข้าใช้งานได้เฉพาะ CMU Account ที่กำหนดไว้เท่านั้น
    \item เว็บแอปพลิเคชันที่สามารถเก็บข้อมูลไฟล์ต่างๆ ได้
\end{enumerate}
\section{\ifenglish Expected outcomes\else ประโยชน์ที่ได้รับ\fi}
\begin{enumerate}
    \item ผู้ใช้สามารถจัดเก็บข้อมูลได้อย่างมีประสิทธิภาพ
    \item ผู้ใช้สามารถเข้าถึงข้อมูลได้อย่างสะดวก รวดเร็ว
    \item ผู้ใช้สามารถนำข้อมูลมาสรุปผลตามเวลาที่เลือกได้
\end{enumerate}
\section{\ifenglish Technology and tools\else เทคโนโลยีและเครื่องมือที่ใช้\fi}
\begin{itemize}
    \item ภาษาโปรแกรมมิ่ง: Typescript
    \item Front-end: Next.js, React, Tailwind CSS
    \item Back-end: NestJs, Prisma
    \item ฐานข้อมูล: PostgreSQL
    \item Object Storage: MinIO
    \item เทคโนโลยีและเครื่องมืออื่นๆ: Docker, Figma, Github
\end{itemize}    
% \subsection{\ifenglish Hardware technology\else เทคโนโลยีด้านฮาร์ดแวร์\fi}

% \subsection{\ifenglish Software technology\else เทคโนโลยีด้านซอฟต์แวร์\fi}

\section{\ifenglish Project plan\else แผนการดำเนินงาน\fi}

\begin{plan}{1}{2025}{4}{2026}
    \planitem{1}{2025}{2}{2025}{Tools Research}
    \planitem{1}{2025}{3}{2025}{Requirements}
    \planitem{5}{2025}{7}{2025}{Design}
    \planitem{7}{2025}{7}{2025}{Set up Environments}
    \planitem{8}{2025}{10}{2025}{Sprint 1: ทำระบบจัดเก็บข้อมูล}
    \planitem{11}{2025}{1}{2026}{Sprint 2: ทำระบบ login}
    \planitem{2}{2026}{4}{2026}{Sprint 3: ทำระบบสรุปข้อมูล}
    \planitem{3}{2026}{4}{2026}{Report}
\end{plan}

\newpage

\section{\ifenglish Roles and responsibilities\else บทบาทและความรับผิดชอบ\fi}
\begin{enumerate}
\item นายกิตปกรณ์ ทองโคตร รับหน้าที่ในการทำ Components ของ Front-end, Initialize Project โดยใช้ Github, วางโครงสร้างของระบบ ทั้ง Front-end และ Back-end
\item นายนรภัทร จินดาสุ่น รับหน้าที่ในการนัดหมายเพื่อพูดคุยกับ Stakeholder, ออกแบบระบบโดยใช้ Figma, จัดวางระบบ Front-end, ติดตามผลการทำงานของเพื่อนในกลุ่ม
\item นายปัณณวิชญ์ เศรษฐสิริวาณิช รับหน้าที่ในการทำระบบฐานข้อมูลทั้งหมด, ทำระบบ Back-end
\end{enumerate}

\section{\ifenglish%
Impacts of this project on society, health, safety, legal, and cultural issues
\else%
ผลกระทบด้านสังคม สุขภาพ ความปลอดภัย กฎหมาย และวัฒนธรรม
\fi}

เมื่อนักสังคมสงเคราะห์สามารถทำงานได้สะดวก รวดเร็วขึ้น ทางผู้รับบริการก็สามารถได้รับบริการได้ดีขึ้นทั้งทางตรงและทางอ้อม เช่นการบริการที่เร็วขึ้น ทำให้รับผู้รับบริการได้มากขึ้น และการสรุปผลข้อมูลก็สามารถลดภาระของนักสังคมสงเคราะห์ที่ต้องไปสรุปข้อมูลเองด้วย
