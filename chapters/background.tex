\chapter{\ifenglish Background Knowledge and Theory\else ทฤษฎีที่เกี่ยวข้อง\fi}

\section{หน้าที่ของนักสังคมเคราะห์}

นักสังคมสงเคราะห์มีหน้าที่สำคัญในการช่วยเหลือและสนับสนุนบุคคล ครอบครัว และชุมชนที่เผชิญปัญหาทางสังคม อารมณ์ หรือเศรษฐกิจ เพื่อให้สามารถปรับตัวและดำเนินชีวิตในสังคมได้อย่างเหมาะสม โดยมีบทบาทที่ครอบคลุมหลายด้าน ซึ่งแต่ละบทบาทล้วนมีความสำคัญต่อคุณภาพชีวิตของผู้ที่ต้องการความช่วยเหลือ

\subsection{ให้คำปรึกษาและแนะแนวทางช่วยเหลือ}

นักสังคมสงเคราะห์ให้คำปรึกษาโดยรับฟัง ประเมินปัญหา และให้แนวทางแก้ไข พร้อมแนะนำสิทธิและสวัสดิการที่เหมาะสม เพื่อให้ผู้ที่ต้องการความช่วยเหลือได้รับการดูแลอย่างเหมาะสม
\subsection{ประสานงานและส่งต่อหน่วยงานที่เกี่ยวข้อง}

นักสังคมสงเคราะห์ทำงานร่วมกับบุคลากรด้านสุขภาพและสวัสดิการเพื่อให้ผู้ป่วยได้รับการดูแลครบถ้วน พวกเขาประสานงานกับหน่วยงานต่าง ๆ เพื่อจัดหาความช่วยเหลือเพิ่มเติม เช่น ที่พักพิง การสนับสนุนทางกฎหมาย และการเข้าถึงบริการทางสังคม

\subsection{ช่วยเหลือกลุ่มเปราะบางในสังคม}

นักสังคมสงเคราะห์ช่วยเหลือกลุ่มเปราะบาง เช่น เด็กถูกทอดทิ้ง ผู้สูงอายุไร้ผู้ดูแล ผู้พิการ และคนไร้บ้าน โดยจัดหาทรัพยากรที่จำเป็น เช่น อาหาร ที่พัก และเงินสนับสนุน พร้อมให้ความรู้เพื่อพัฒนาคุณภาพชีวิตและส่งเสริมการพึ่งพาตนเอง
\subsection{ติดตามผลและประเมินสถานการณ์}

นักสังคมสงเคราะห์ติดตามผลการช่วยเหลืออย่างต่อเนื่อง เพื่อให้แน่ใจว่าตรงกับความต้องการของแต่ละบุคคล หากพบปัญหาเพิ่มเติม พวกเขาจะปรับแผนให้เหมาะสม พร้อมจัดทำรายงานเพื่อใช้พัฒนาระบบสวัสดิการในอนาคต

\newpage
\section{ด้านเทคโนโลยีที่ใช้พัฒนาเว็บแอปพลิเคชัน}
\subsection{Typescript}
TypeScript เป็นภาษาโปรแกรมที่พัฒนาโดย Microsoft ซึ่งเพิ่มระบบตรวจสอบ Type ให้กับ JavaScript ทำให้สามารถควบคุมและจัดการ Type ในโปรแกรมได้อย่างมีประสิทธิภาพ ช่วยลดข้อผิดพลาดที่อาจเกิดขึ้นก่อนรันโปรแกรม ซึ่งเป็นข้อได้เปรียบสำคัญในการพัฒนาแอปพลิเคชันขนาดใหญ่ เนื่องจากใน JavaScript ข้อผิดพลาดเกี่ยวกับ Type มักจะปรากฏขึ้นในระหว่างการทำงานของโปรแกรม \cite{aiw}

\begin{figure}[H]
  \begin{center}
  \includegraphics[width=0.2\textwidth]{typescript_original_logo_icon_146317.png}
  \end{center}
  \caption[Typescript]{Typescript}
\end{figure}
\subsection{React}
React เป็นไลบรารี JavaScript ที่พัฒนาโดย Facebook โดยถูกออกแบบมาเพื่อช่วยในการสร้าง User Interface (UI) ที่มีประสิทธิภาพและยืดหยุ่นสูง โครงสร้างหลักของ React คือแนวคิดเกี่ยวกับ Component-Based Architecture ซึ่งช่วยให้การพัฒนา UI เป็นไปอย่างเป็นระบบ แต่ละ Component สามารถนำกลับมาใช้ซ้ำได้ React ใช้แนวคิด Virtual DOM เพื่อเพิ่มประสิทธิภาพในการอัปเดต UI โดยการเปรียบเทียบความเปลี่ยนแปลงก่อนที่จะเรนเดอร์จริง ทำให้การทำงานรวดเร็วและลดภาระของเบราว์เซอร์ React ได้รับความนิยมอย่างแพร่หลายในวงการพัฒนาเว็บแอปพลิเคชัน เนื่องจากช่วยให้การจัดการ UI ซับซ้อนทำได้ง่ายขึ้น และสามารถรองรับการพัฒนาแบบไดนามิกได้เป็นอย่างดี \cite{aiw2}
\begin{figure}[H]
  \begin{center}
  \includegraphics[width=0.2\textwidth]{1174949_js_react js_logo_react_react native_icon.png}
  \end{center}
  \caption[React]{React}
\end{figure}

\newpage

\subsection{Next.js}
Next.js เป็น React Web Framework ที่ออกแบบมาเพื่อให้การพัฒนาเว็บแอปพลิเคชันสะดวกขึ้น โดยมีการตั้งค่าและการคอนฟิกหลายๆ อย่างให้พร้อมใช้งาน เช่น การไม่ต้องตั้งค่าใดๆ เอง (Zero Config) และสามารถใช้งานได้ทันทีในสภาพแวดล้อมการผลิต (Ready for Production) นอกจากนี้ Next.js รองรับการทำ SEO ได้ดีกว่า React ปกติ เนื่องจากเป็นการเรนเดอร์แบบ Server Side Rendering (SSR) ซึ่งแตกต่างจาก React ที่ส่วนใหญ่ใช้ Client Side Rendering ในการทำงาน นอกจากนี้ยังรองรับฟีเจอร์เช่น Code Splitting, Static Site Generation (SSG), การสร้าง Dynamic Page, การสร้าง APIs และการปรับแต่งเซิร์ฟเวอร์ได้ รวมถึงมีเอกสารประกอบที่อ่านเข้าใจง่ายและมีตัวอย่างให้ศึกษา ทำให้ผู้ที่เคยใช้งาน React สามารถเริ่มใช้งาน Next.js ได้อย่างรวดเร็วและไม่ยาก \cite{aiw3}
\begin{figure}[H]
  \begin{center}
  \includegraphics[width=0.2\textwidth]{next-js-seeklogo.png}
  \end{center}
  \caption[Next.js]{Next.js}
\end{figure}
\subsection{Nest JS}
Nest JS เป็น Framework สำหรับพัฒนาแอปพลิเคชันฝั่งเซิร์ฟเวอร์ที่สร้างขึ้นบน Node.js โดยออกแบบมาให้มีประสิทธิภาพและยืดหยุ่นสูง โครงสร้างของ Nest JS ถูกออกแบบให้ใช้ TypeScript เป็นหลัก (แม้ว่าจะรองรับ JavaScript ด้วย) และยึดแนวคิดเชิง Object-Oriented Programming (OOP) อย่างสมบูรณ์ ซึ่งช่วยให้การจัดระเบียบโค้ดเป็นไปอย่างเป็นระบบและมีโครงสร้างที่ชัดเจน นอกจากนี้ NestJS ยังสามารถทำงานร่วมกับไลบรารีอื่นๆ ได้อย่างง่ายดาย เช่น Express.js และ Socket.io ทำให้สามารถขยายขีดความสามารถของแอปพลิเคชันได้อย่างยืดหยุ่นและมีประสิทธิภาพ \cite{aiw4}
\begin{figure}[H]
  \begin{center}
  \includegraphics[width=0.2\textwidth]{nest-js-icon.png}
  \end{center}
  \caption[Nest JS]{Nest JS}
\end{figure}

\newpage

\subsection{PostgreSQL}
PostgreSQL เป็นระบบจัดการฐานข้อมูลเชิงสัมพันธ์แบบโอเพ่นซอร์สระดับองค์กรที่ทรงพลัง รองรับทั้งการจัดเก็บข้อมูลแบบเชิงสัมพันธ์ผ่าน SQL และข้อมูลแบบไม่เชิงสัมพันธ์ผ่าน JSON โดดเด่นด้วยความน่าเชื่อถือ ความปลอดภัย และความแม่นยำในระดับสูง อีกทั้งยังได้รับการสนับสนุนจากชุมชนผู้พัฒนาที่แข็งแกร่ง ด้วยประสิทธิภาพและความยืดหยุ่นของ PostgreSQL ทำให้เป็นตัวเลือกหลักสำหรับเว็บแอปพลิเคชัน อุปกรณ์พกพา และโซลูชันด้านการวิเคราะห์ข้อมูลเชิงพื้นที่ที่ต้องการระบบฐานข้อมูลที่มีเสถียรภาพสูง \cite{aiw6}
\begin{figure}[H]
  \begin{center}
  \includegraphics[width=0.2\textwidth]{PostgreSQL-Logo.wine.png}
  \end{center}
  \caption[PostgreSQL]{PostgreSQL}
\end{figure}
\subsection{MinIO}
MinIO เป็นระบบจัดเก็บข้อมูลแบบ Object Storage ที่รองรับการจัดเก็บข้อมูลประเภทต่างๆ เช่น ไฟล์ไบนารี, ไฟล์เสียง, รูปภาพ, วิดีโอ และสเปรดชีต โดยผู้ใช้สามารถเข้าถึงและจัดการข้อมูลที่เก็บอยู่ใน MinIO ผ่าน REST API ทำให้สามารถเชื่อมต่อกับแอปพลิเคชันหรือระบบต่างๆ ได้อย่างสะดวกและมีประสิทธิภาพ \cite{aiw5}
\begin{figure}[H]
  \begin{center}
  \includegraphics[width=0.2\textwidth]{cdnlogo.com_minio.png}
  \end{center}
  \caption[MinIO]{MinIO}
\end{figure}
\subsection{Docker}
Docker เป็นแพลตฟอร์มซอฟต์แวร์ที่ช่วยให้การสร้าง ทดสอบ และปรับใช้แอปพลิเคชันเป็นไปอย่างรวดเร็วและมีประสิทธิภาพ โดยใช้ คอนเทนเนอร์ (Containers) ซึ่งเป็นหน่วยบรรจุซอฟต์แวร์ที่รวมทุกองค์ประกอบที่จำเป็นสำหรับการทำงาน เช่น ไลบรารี เครื่องมือระบบ โค้ด และรันไทม์ ทำให้สามารถรันแอปพลิเคชันได้อย่างเสถียรในทุกสภาพแวดล้อม ด้วยความสามารถนี้ Docker ช่วยให้การปรับใช้และขยายขนาดแอปพลิเคชันเป็นไปอย่างราบรื่นและมั่นใจได้ว่าโค้ดจะทำงานได้อย่างสม่ำเสมอในทุกระบบ \cite{aiw7}
\begin{figure}[H]
  \begin{center}
  \includegraphics[width=0.2\textwidth]{4373190_docker_logo_logos_icon.png}
  \end{center}
  \caption[Docker]{Docker}
\end{figure}
\subsection{Tailwind CSS}
Tailwind CSS เป็นหนึ่งใน CSS Framework ที่ออกแบบมาในแนวทาง Utility-first ซึ่งหมายความว่าแต่ละคลาสใน Tailwind จะกำหนดคุณสมบัติของ CSS เพียงหนึ่งอย่างโดยเฉพาะ เช่น สีพื้นหลัง (background-color), ขอบ (border), ขนาดตัวอักษร (font-size) และอื่น ๆ แทนที่จะเป็นคลาสที่รวมชุดคำสั่งสำเร็จรูปไว้

Tailwind จึงเหมาะสำหรับนักพัฒนาที่ต้องการความยืดหยุ่นสูงในการออกแบบหน้าเว็บ เนื่องจากสามารถกำหนดสไตล์ขององค์ประกอบต่าง ๆ ได้อย่างแม่นยำโดยไม่ต้องพึ่งพาคลาสคอมโพเนนต์สำเร็จรูป ซึ่งมักมีข้อจำกัดในการปรับแต่ง นอกจากนี้ การใช้ Tailwind ยังช่วยให้การจัดการโค้ด CSS เป็นระบบมากขึ้น ลดความซับซ้อนของไฟล์สไตล์ชีต และส่งเสริมแนวทางการพัฒนาเว็บที่มีประสิทธิภาพ \cite{aiw8}
\begin{figure}[H]
  \begin{center}
  \includegraphics[width=0.2\textwidth]{pngwing.com (4).png}
  \end{center}
  \caption[Tailwind CSS]{Tailwind CSS}
\end{figure}

\subsection{Prisma}
Prisma ORM เป็นเครื่องมือสำหรับจัดการฐานข้อมูลในแอปพลิเคชัน TypeScript และ JavaScript ที่ช่วยให้การทำงานกับฐานข้อมูลมีประสิทธิภาพและใช้งานง่ายขึ้น

Prisma มีเครื่องมือสำหรับ จัดการโครงสร้างฐานข้อมูล (schema management), ทำ migrations และเรียกใช้งานข้อมูลผ่าน Prisma Client ซึ่งช่วยลดความซับซ้อนของ SQL Query รองรับฐานข้อมูลยอดนิยม เช่น PostgreSQL, MySQL, SQLite และ MongoDB ทำให้เป็นตัวเลือกที่เหมาะสำหรับนักพัฒนาที่ต้องการความยืดหยุ่นในการจัดการข้อมูล \cite{aiw9}
\begin{figure}[H]
  \begin{center}
  \includegraphics[width=0.2\textwidth]{pngwing.com (3).png}
  \end{center}
  \caption[Prisma]{Prisma}
\end{figure}

\newpage

\section{\ifenglish%
\ifcpe CPE \else ISNE \fi knowledge used, applied, or integrated in this project
\else%
ความรู้ตามหลักสูตรซึ่งถูกนำมาใช้หรือบูรณาการในโครงงาน
\fi
}
\begin{itemize}

  \item[-] \textbf{Fundamentals of Database Systems (261342)} ศึกษาหลักการออกแบบและการใช้งานฐานข้อมูล โดยใช้ MySQL ในการสร้าง เชื่อมต่อ และจัดการฐานข้อมูล รวมถึงการเขียนคำสั่ง SQL เพื่อจัดการข้อมูลภายในฐานข้อมูล
  \item[-] \textbf{Software Engineering (261361)} ศึกษากระบวนการทางวิศวกรรมซอฟต์แวร์ ตั้งแต่การเก็บรวบรวมความต้องการของผู้ใช้ การกำหนดเป้าหมายของระบบ การวางแผนการทำงาน การประเมินและทดสอบระบบ ไปจนถึงการปรับปรุงและแก้ไขระบบให้มีประสิทธิภาพ
  \item[-] \textbf{Basic CPE Lab (261207)} ฝึกปฏิบัติการพัฒนาเว็บแอปพลิเคชัน โดยประยุกต์ใช้ความรู้ด้าน HTML สำหรับการพัฒนา Front-End และ Python สำหรับการพัฒนา Back-End รวมถึงการออกแบบ UX/UI เพื่อให้ระบบใช้งานได้อย่างมีประสิทธิภาพ
  \item[-] \textbf{Selected Topics in Computer Engineering (261494)} ศึกษาและประยุกต์ใช้เทคโนโลยี Docker และ Cloud ในการสร้างระบบจำลองสำหรับพัฒนาเว็บแอปพลิเคชัน
  \item[-] \textbf{Object-Oriented Programming (261200)} ศึกษาหลักการเขียนโปรแกรมเชิงวัตถุ (OOP) และการใช้ Model-View-Controller (MVC) รวมถึงการใช้ Web socket ในการออกแบบและพัฒนาเว็บแอปพลิเคชัน

\end{itemize}

% \section{\ifenglish%
% Extracurricular knowledge used, applied, or integrated in this project
% \else%
% ความรู้นอกหลักสูตรซึ่งถูกนำมาใช้หรือบูรณาการในโครงงาน
% \fi
% }

% อธิบายถึงความรู้ต่างๆ ที่เรียนรู้ด้วยตนเอง และแนวทางการนำความรู้เหล่านั้นมาใช้ในโครงงาน
